%------------------------------------------------------------------------------
\documentclass[a4paper,10pt]{article}
%------------------------------------------------------------------------------

\usepackage{amssymb}
\usepackage{amsmath}

% Some math operators
\DeclareMathOperator{\Div}{div}
\DeclareMathOperator{\Grad}{grad}
\newcommand{\inner}[2]{\langle #1, #2 \rangle}
\newcommand{\R}{\mathbb{R}}
\newcommand{\foralls}{\forall\,}
\newcommand{\ddt}[1]{{#1}_t}
\newcommand{\dx}{\dif{}x}
\newcommand{\ds}{\dif{}s}
\newcommand{\dS}{\dif{}S}

\begin{document}

We want to solve a dynamic version of the cardiac mechanics equations, including the
inertia term that is usually neglected. The boudnary conditions are as follows:
\begin{itemize}
\item A standard pressure term on the endocardium ($\Gamma_{endo}$)
\item Normal forces on the epicardium ($\Gamma_{endo}$) are a combination of a linear spring
and linear friction
\item Zero tangent force on the epicardium
\item The traction (tangent + normal) on the base ($\Gamma_{top}$) is the sum of
a linear spring and linear friction.
\end{itemize}
In the benchmark description, the equations are written as
\begin{align*}
  \rho \ddot{u} - \nabla\cdot(J\sigma F^{-T}) &= 0, \mbox{ in } \Omega ,\\
  \sigma J F^{-T}N &= pJF^{-TN}, \mbox{ on } \Gamma_{endo}, \\
  \sigma JF^{-T}N\cdot N + \alpha{epi}u\cdot N + \beta_{epi}\dot{u}\cdot N &= 0, \mbox{ on }  \Gamma_{epi}, \\
  \sigma JF^{-T}N\times N &=0, \mbox{ on }  \Gamma_{epi}, \\
  \sigma JF^{-T}N + \alpha_{top}u + \beta_{top}\dot{u} &= 0, \mbox{ on } \Gamma_{top},
\end{align*}
where we have used the notation $\sigma$ for the Cauchy stress to be more in line with standard notation.

In terms of the first Piola-Kirchhoff stress, given by $P = J\sigma F^{-T}$, the problem reads
\begin{align}
  \rho \ddot{u} - \nabla\cdot P &= 0, \mbox{ in } \Omega ,\label{dyn_eq0}\\
  PN &= pJF^{-TN}, \mbox{ on } \Gamma_{endo}, \\
  PN\cdot N + \alpha_{epi}u\cdot N + \beta_{epi}\dot{u}\cdot N &= 0, \mbox{ on }  \Gamma_{epi}, \\
  PN\times N &=0, \mbox{ on }  \Gamma_{epi}, \\
  PN + \alpha_{top}u + \beta_{top}\dot{u} &= 0, \mbox{ on } \Gamma_{top}.\label{dyn_bc3}
\end{align}

If we introduce the notation $a= \ddot{u}, v=\dot{u}$
for the acceleration and velocity, respectively, a weak form
of \eqref{dyn_eq0}-\eqref{dyn_bc3} can be written as
\begin{equation}
\begin{aligned}
\int_{\Omega} \rho a \cdot w \, dX + \int_{\Omega} P:Grad(w) \, dX -\int_{\Gamma_{endo}} p I JF^{-T}N \cdot w \, dS  \\
+\int_{\Gamma_{epi}} \big(\alpha_{epi} u \cdot N + \beta_{epi} v \cdot N \big) w \cdot N \, dS  \\
+\int_{\Gamma_{top}} \alpha_{top} u \cdot w + \beta_{top} v \cdot w \, dS = 0 \quad \forall w \in H^1(\Omega).
\end{aligned}\label{weak1}
\end{equation}
In order to integrate \eqref{weak1} in time, we need to express $a$ and $v$ in
terms of the displacement $u$. This can be done by numerous methods, a few of
which will be discussed below.

\section{Time integration schemes}
\subsection{The Newmark $\beta$ scheme}
A popular method for time integration is the Newmark $\beta$ scheme,
which gives $a,v$ as
\begin{align}
v_{i+1} &= v_i + (1-\gamma) \Delta t a_i + \gamma \Delta t a_{i+1}, \label{N_v}\\
a_{i+1} &= \frac{u_{i+1} - (u_i + \Delta t v_i + (0.5 - \beta) \Delta t^2 a_i)}{\beta \Delta t^2}, \label{N_a}
\end{align}
Inserting these expressions into \eqref{weak1} and assuming $u_i,v_i,a_i$ known,
we obtain the time discrete weak form
\begin{equation}
\begin{aligned}
\int_{\Omega} \rho a_{i+1} \cdot w \, dX + \int_{\Omega} P(u_{i+1},v_{i+1}):Grad(w) \, dX -\int_{\Gamma_{endo}} p I JF_{i+1}^{-T}N \cdot w \, dS  \\
+\int_{\Gamma_{epi}} \big(\alpha_{epi} u_{i+1} \cdot N + \beta_{epi} v_{i+1} \cdot N \big) w \cdot N \, dS  \\
+\int_{\Gamma_{top}} \alpha_{top} u_{i+1} \cdot w + \beta_{top} v_{i+1} \cdot w \, dS = 0 \quad \forall w \in H^1(\Omega),
\end{aligned}\label{weak2}
\end{equation}
where $F_{i+1} = I + \nabla u_{i+1}$ and $v_i,a_i$ are given by
\eqref{N_v}-\eqref{N_a}.\footnote{We could insert these expressions into \eqref{weak2} to eliminate $v_i,a_i$
and yield a non-linear weak form with $u_{i+1}$ as the only unknown. However, the
resulting form becomes fairly complex, and we avoid this task by implementing \eqref{weak2} and
\eqref{N_v}-\eqref{N_a} directly as UFL forms in the code, and leave the algebra to UFL.}

A common choice is $\alpha=1/2, \beta=1/4$, which yields
the \emph{average constant acceleration (middle point rule)} method, while $\alpha=1/2, \beta=1/6$
yields the \emph{linear acceleration method} where the acceleration is linearly varying between
$t$ and $t+\Delta t$. Both methods are unconditionally stable and second order accurate, but
they introduce no numerical damping and can suffer from spurious oscillations. Choosing $\gamma > 1/2$
introduces energy dissipation which can avoid such oscillations, but reduces the
accuracy to first order. The method is unconditionally stable for $1/2 \leq \gamma \leq 2\beta$.

A summary of solver results for a few selected parameter values is provided in
Table \ref{tab:conv_res}. We have not investigated the source of the
observed divergence, but it may be due to the lack of
numerical dissipation in the method. This is known to lead to overshoot
and spurious oscillations for certain problems, and such oscillations and
overshoots in the displacement can easily lead to divergence of the non-linear
solver. However, since our problem includes damping in the form of a visco-elastic
material, it may seem counter-intuitive that a dissipative scheme is needed for stability.

\begin{table}[htb]
  \begin{center}
    \begin{tabular}{|r|r|l|}
      \hline
      $\alpha$ & $\beta $& Result \\
      \hline \hline
      1/2 & 1/4 &  Diverges for $\Delta t = 0.01, 0.001$ \\ \hline
      1/2 & 1/6 &  Diverges for $\Delta t = 0.01, 0.001$ \\ \hline
      0.6 & 0.3 & Diverges for $\Delta t = 0.01, 0.001$ \\ \hline
      0.8 & 0.45 & Converges, visually ok solution \\ \hline
    \end{tabular}
  \end{center}
  \caption{Summary of selected solver results for the Newmark $\beta$ method. The two
  top rows yield a second order accurate method, while the two lower combinations
  are first order accurate in the displacement.}
  \label{tab:conv_res}
\end{table}

\subsection{The generalized $\alpha$ method}
Several methods have been derived with the purpose of avoiding the non-physical
oscillations sometimes seen in the Newmark method for $\gamma =1/2$, while retaining
the second order convergence in $\Delta t$. One class of such methods is called
the generalized $\alpha$ or G-$\alpha$ methods, which introduce additional parameters
$\alpha_f$ and $\alpha_m$. The methods use the Newmark approximations in
\eqref{N_v}-\eqref{N_a} to approximate $v,a$, but evaluate the terms of the
weak form at times $t_{i+1}-\Delta t\alpha_m$ and  $t_{i+1}-\Delta t\alpha_f$.
Specifically, the inertia term is evaluated at $t_{i+1}-\Delta t\alpha_m$, and the
other terms at $t_{i+1}-\Delta t\alpha_f$. The weak form becomes
\begin{equation}
\begin{aligned}
\int_{\Omega} \rho a_{i+1-\alpha_m} \cdot w \, dX + \int_{\Omega} P_{i+1-\alpha_f}:Grad(w) \, dX -\int_{\Gamma_{endo}} p I JF_{i+1-\alpha_f}^{-T}N \cdot w \, dS  \\
+\int_{\Gamma_{epi}} \big(\alpha_{epi} u_{i+1-\alpha_f} \cdot N + \beta_{epi} v_{i+1-\alpha_f} \cdot N \big) w \cdot N \, dS  \\
+\int_{\Gamma_{top}} \alpha_{top} u_{i+1-\alpha_f} \cdot w + \beta_{top} v_{i+1-\alpha_f} \cdot w \, dS = 0 \quad \forall w \in H^1(\Omega),
\end{aligned}\label{weak3}
\end{equation}
with
\begin{align*}
  u_{i+1-\alpha_f} &= (1-\alpha_f)u_{i+1}-\alpha_f u_i, \\
  v_{i+1-\alpha_f} &= (1-\alpha_f)v_{i+1}-\alpha_f v_i, \\
  a_{i+1-\alpha_m} &= (1-\alpha_m)a_{i+1}-\alpha_m a_i,
\end{align*}
$v_{i+1},a_{i+1}$ given by \eqref{N_v}-\eqref{N_a}, and
\begin{align*}
F_{i+1-\alpha_f} &= I + \nabla u_{i+1-\alpha_f} ,
P_{i+1-\alpha_f} &= P(u_{i+1-\alpha_f}, v_{i+1-\alpha_f}).
\end{align*}
The only difference between \eqref{weak3} and \eqref{weak2} is the time point at which the
terms of the equation are evaluated.

Different choices of the four parameters $\alpha_m, \alpha_f, \beta, \gamma$ yield
methods with different accuracy and stability properties. Tables 1-3 in \cite{erlicher2002analysis}
provides an overview of parameter choices for methods in the literature,
as well as conditions for stability and convergence.
We have used the choice $\alpha_m =0.2, \alpha_f=0.4$, and
\begin{align*}
  \gamma &= 1/2 + \alpha_f-\alpha_m ,\\
  \beta &= \frac{(\gamma + 1/2)^2}{4} .
\end{align*}
For this choice the solver converges through the time interval of interest, and
the convergence is second order.


\paragraph{Alternative formulations of the G-$\alpha$ method.}
\begin{enumerate}
  \item In addition to the choice of the four parameters, different choices
  can be made in how the quantities $F_{i+1-\alpha_f}$ and $P_{i+1-\alpha_f}$
  are approximated. Alternative choices include
  \begin{align*}
   F_{i+1-\alpha_f} = (1-\alpha_f) (I + \nabla u_{i+1}) + \alpha_f(I + \nabla u_{i}), \\
   P_{i+1-\alpha_f} = (1-\alpha_f) P(u_{i+1}, v_{i+1}) + \alpha_f P (I + \nabla u_{i}, v_{i}) ,
 \end{align*}
 and other options, see, for instance, \cite{erlicher2002analysis} and references therein.
 These alternatives have not yet been explored.

 \item So far, we have implemented the weak form \eqref{weak3} and the relations
 \eqref{N_v}-\eqref{N_a} directly, and relied on UFL to handle the algebra to turn it
 into a problem with $u_{i+1}$ as the only unknown. This seems to work well, but
 there are alternatives. For instance, Eq. (14) in \cite{erlicher2002analysis} is
 a relatively compact equation for $u_{i+1}$ based on the G-$\alpha$ method. It should
 be possible to derive a weak form similar to this equation for our problem, which could be
 solved and compared with the UFL approach.
\end{enumerate}


\appendix
\section{The Newmark $\beta$ scheme}
We consider first a general dynamic equation on the form
\[
M\ddot{u} + C\dot{u} + f^{int}(u) = f_{ext},
\]
where $M$ is the mass matric, $C$ is the damping matrix, and $f^{int}, f^{ext}$
are internal and external forces. Note that the mechanics equation above can also
be written on this general form, since the stress tensor $P$ includes a viscous
component which is a linear function of the velocity $\dot{u}$.

The Newmark $\beta$ method
is then commonly written as
\begin{align}
v_{i+1} &= v_i + (1-\gamma) \Delta t a_i + \gamma \Delta t a_{i+1}, \\
a_{i+1} &= \frac{u_{i+1} - (u_i + \Delta t v_i + (0.5 - \beta) \Delta t^2 a_i)}{\beta \Delta t^2}, \\
Ma_{i+1} &+ C v_{i+1} + f^{int}(u_{i+1}) = f^{ext}_{i+1} .
\end{align}

\section{Computing the cavity volume}
One of the quantities reported for the benchmark problem is the cavity volume.
The simplest way to compute this quantity is by an integral over the cavity, i.e.
by
\begin{align}
V &= \int_\Omega dx = \int_{\Omega_0} J dX,
\label{vol0}\end{align}
where $J=\det(F)$ and $F$ is the deformation gradient, or,
equivalently, the Jacobian of the coordinate transformation from $X$
to $x$. This is the approach described in the benchmark description.
The $F$ and $J$ fields inside the cavity are computed from a displacement field
constructed by harmonic lifting.

Our approach for computing the volume does not require a mesh of the cavity,
but instead applies the divergence theorem to turn the volume integral into
a surface integral over the endocardial surface. The divergence theorem states
that
\[
\int_\Omega \nabla\cdot f dx = \int_{\partial\Omega} f\cdot \mathbf{n} ds
\]
holds for any smooth function $f$ on a volume $\Omega$, with $\mathbf{n}$ being the
outward surface normal of $\partial\Omega$. If we introduce an
arbitrary vector-valued function $\mathbf{g}$ satisfying $\nabla\cdot \mathbf{g} = 1$, we have
\[
V =\int_{\Omega} dx = \int_{\Omega}\nabla\cdot \mathbf{g}dx
= \int_{\partial\Omega}\mathbf{g}\cdot\mathbf{n}ds .
\]
There are many simple choices for the function $\mathbf{g}$, for instance $\mathbf{g}=(x_1,0,0)$ or
$\mathbf{g}=\mathbf{x}/3$.

One limitation of this approach is that the surface integral is performed over the
entire boundary $\partial\Omega$, i.e., a closed surface, while the cavity volumes
of interest are usually open at the base. However, if we know the characteristics
of the geometry it is often possible to choose the function $g$ so that the
contribution from the missing part of the surface is zero, and in this case the
volume calculation is accurate. For the benchmark problem, the open
part of the surface is a plane with surface normal $(1,0,0)$. The plane moves
in the $x_1$ direction, but the surface normal stays constant through the
deformation. We can choose, for instance, $\mathbf{g}=(0,x_2,0)$ to ensure that the
contribution from this part of the surface is always zero. We then have
\begin{align}
V &= \int_{\partial\Omega_{endo}}(0,x_2,0) \cdot\mathbf{n}ds ,
\label{vol1}\end{align}
where $\partial\Omega_{endo}$ is the deformed configuration of the (open)
endocardial surface.


We want to convert the surface integral in \eqref{vol1} to an
integral in the reference configuration. We have
\begin{align}
\mathbf{u}(\mathbf{X},t) &= \mathbf{x}(\mathbf{X},t)-\mathbf{X},
\label{displ}\end{align}
where $\mathbf{X}$ is the position vector in the undeformed (material) configuration.
From Nanson's formula we have that
\begin{align}
\mathbf{n}ds &= J\mathbf{F}^{-T}\mathbf{\eta}dS,
\label{nanson}\end{align}
where $\mathbf{\eta}, dS$ are the surface normal and area of the undeformed surface.
Using \eqref{displ}-\eqref{nanson}, we can convert \eqref{vol1} to
\begin{align}
V &=  \int_{\partial\Omega_{endo,0}}(u_2+X_2)\cdot
J\mathbf{F}^{-T}\mathbf{\eta}dS ,
\label{vol2}\end{align}
where $\partial\Omega_{endo,0}$ is the undeformed configuration
of the endocardial surface.

\bibliographystyle{plain}
\bibliography{./references}





\end{document}
