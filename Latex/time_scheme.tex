%------------------------------------------------------------------------------
\documentclass[a4paper,10pt]{article}
%------------------------------------------------------------------------------

\usepackage{amssymb}
\usepackage{amsmath}

% Some math operators
\DeclareMathOperator{\Div}{div}
\DeclareMathOperator{\Grad}{grad}
\newcommand{\inner}[2]{\langle #1, #2 \rangle}
\newcommand{\R}{\mathbb{R}}
\newcommand{\foralls}{\forall\,}
\newcommand{\ddt}[1]{{#1}_t}
\newcommand{\dx}{\dif{}x}
\newcommand{\ds}{\dif{}s}
\newcommand{\dS}{\dif{}S}

\begin{document}

We want to solve a dynamic version of the cardiac mechanics equations, including the
inertia term that is usually neglected. The bounary conditions are as follows:
\begin{itemize}
\item A standard pressure term on the endocardium ($\Gamma_{endo}$)
\item Normal forces on the epicardium ($\Gamma_{endo}$) are a combination of a linear spring
and linear friction
\item Zero tangent force on the epicardium
\item The traction (tangent + normal) on the base ($\Gamma_{top}$) is the sum of
a linear spring and linear friction.
\end{itemize}
In the benchmark description, the equations are written as
\begin{align*}
  \rho \ddot{u} - \nabla\cdot(J\sigma F^{-T}) &= 0, \mbox{ in } \Omega ,\\
  \sigma J F^{-T}N &= pJF^{-TN}, \mbox{ on } \Gamma_{endo}, \\
  \sigma JF^{-T}N\cdot N + \alpha{epi}u\cdot N + \beta_{epi}\dot{u}\cdot N &= 0, \mbox{ on }  \Gamma_{epi}, \\
  \sigma JF^{-T}N\times N &=0, \mbox{ on }  \Gamma_{epi}, \\
  \sigma JF^{-T}N + \alpha_{top}u + \beta_{top}\dot{u} &= 0, \mbox{ on } \Gamma_{top},
\end{align*}
where we have used the notation $\sigma$ for the Cauchy stress to be more in line with standard notation.

In terms of the first Piola-Kirchhoff stress, given by $P = J\sigma F^{-T}$, the problem reads
\begin{align*}
  \rho \ddot{u} - \nabla\cdot P &= 0, \mbox{ in } \Omega ,\\
  \sigma PN &= pJF^{-TN}, \mbox{ on } \Gamma_{endo}, \\
  \sigma PN\cdot N + \alpha{epi}u\cdot N + \beta_{epi}\dot{u}\cdot N &= 0, \mbox{ on }  \Gamma_{epi}, \\
  \sigma PN\times N &=0, \mbox{ on }  \Gamma_{epi}, \\
  \sigma PN + \alpha_{top}u + \beta_{top}\dot{u} &= 0, \mbox{ on } \Gamma_{top}.
\end{align*}

\section{The Newmark $\beta$ scheme}
We consider first a general dynamic equation on the form
\[
M\ddot{u} + C\dot{u} + f^{int}(u) = f_{ext},
\]
where $M$ is the mass matric, $C$ is the damping matrix, and $f^{int}, f^{ext}$
are internal and external forces. Note that the mechanics equation above can also
be written on this general form, since the stress tensor $P$ includes a viscous
component which is a linear function of the velocity $\dot{u}$.
The mechanics equation above can be written on this form, since the second

We introduce the notation $a= \ddot{u}, v=\dot{u}$
for the acceleration and velocity, respectively. The Newmark $\beta$ method can then
be written as
\begin{align}
v_{i+1} &= v_i + (1-\gamma) \Delta t a_i + \gamma \Delta t a_{i+1}, \\
a_{i+1} &= \frac{u_{i+1} - (u_i + \Delta t v_i + (0.5 - \beta) \Delta t^2 a_i)}{\beta \Delta t^2}, \\
Ma_{i+1} &+ C v_{i+1} + f^{int}(u_{i+1}) = f^{ext}_{i+1}.
\end{align}





\end{document}
